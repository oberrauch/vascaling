\documentclass[margin=10pt]{standalone}
    \usepackage[T1]{fontenc}
    \usepackage[utf8]{inputenc}

    \usepackage{tikz}
        \usetikzlibrary{backgrounds,calc,calendar}
    \usepackage{pgfgantt}
    \usepackage{pgfcalendar}

    \usepackage{calc}
    \usepackage{ifthen}

    %\newcommand{\datenumber}[1]{% to compute the number of days between \startdate and a given date. Unused here
    %   \pgfcalendardatetojulian{#1}{\dtnmbr}%
    %   \advance\dtnmbr by -\mystrtdt%
    %   \the\dtnmbr%
    %}

    % define start and end dates
    \def\startdate{2020-4-16}
    \def\enddate{2020-5-3}
    
    % get number representation of dates
    \newcount\startdatecount
    \newcount\enddatecount
    \pgfcalendardatetojulian{\startdate}{\startdatecount}
    \pgfcalendardatetojulian{\enddate}{\enddatecount}
    
    % compute number of days
    \newcount\numberofdays
    \numberofdays=\enddatecount
    \advance\numberofdays by -\startdatecount

    % compute number of weeks
    \newcount\mynumberofweeks
        \mynumberofweeks\numberofdays\relax
        \advance \mynumberofweeks by -1\relax
        \divide \mynumberofweeks by 7\relax

    % include first/last day in count
    \advance\numberofdays by 1

    % define first week day
    \newcount\firstweekday
        \pgfcalendarjuliantoweekday{\startdatecount}{\firstweekday}

    % compute shift of first week
    \newcount\firstweekendshift
        \firstweekendshift 5\relax
        \advance\firstweekendshift by -\firstweekday\relax
        % if the first day is a Sunday
        \ifnum \firstweekendshift=-1 
            % the first full weekend will thus begin one week after
            \advance \firstweekendshift by 7\relax
        \fi

    % define units
    \def\myxunit{2cm} % width of 1 day
    \def\myyunittitle{.5cm} % height of 1 title line
    \def\myyunitchart{1cm} % height of 1 chart line

    % define the name of weekdays + formatting
    \def\pgfcalendarweekdayletter#1{
        \ifcase#1 Mo\or Th\or We\or Th\or Fr\or
        \textcolor{red!50!white}{Sa}\or \textcolor{red}{\textbf{So}}\fi
    }

    % define function that prints weekday and day of month
    \newcount\currentdatecount
        \currentdatecount=\startdatecount
        \newcount\weekday
    \protected\def\zzz{{\pgfcalendarjuliantoweekday{\currentdatecount}{\weekday}\pgfcalendarweekdayletter{\weekday},
        \pgfcalendarjuliantodate{\currentdatecount}{\year}{\month}{\day}\day. \global\advance\currentdatecount by 1}}
    

\begin{document}

    \begin{ganttchart}[
            hgrid,
            vgrid,
            x unit = \myxunit,
            y unit title = \myyunittitle,
            title height = .75,
            y unit chart = \myyunitchart,
            time slot format = isodate,
            canvas/.append style = {fill opacity=.1},
        ]{\startdate}{\enddate}

            % Title lines
            \gantttitle{\textbf{Expose}}{\numberofdays}\\
            \gantttitlecalendar{month=name} \\
            \gantttitlelist[
                    title list options={var=\y, evaluate=\y as \x using "\zzz"}
                ]{1,...,\numberofdays}{1} \\
            
            % set number of title lines
            \def\numbttitlelines{3} 

            % Chart lines
            \ganttbar{Reserach Question}{2020-4-17}{2020-4-21} \\
            \ganttbar{Methodology}{2020-4-20}{2020-4-25} \\
            \ganttbar{State of the art}{2020-4-23}{2020-4-27} \\
            \ganttbar{Introduction}{2020-4-25}{2020-4-30}
            
            % set number if chart lines
            \def\numbtchartlines{4}

            % display week-ends
            \begin{scope}[on background layer]
                \foreach \i in {0,...,\the\mynumberofweeks}
                \fill [red!10]
                    % on the x-absis: shift the number of days before the first Saturday + 7days*\i;
                    % on the y-absis: shift down the number of title lines * their height
                    (\firstweekendshift*\myxunit+\i*7*\myxunit,-\numbttitlelines*\myyunittitle)
                    rectangle 
                    % on the x-absis: add two days (the week-end);
                    % on the y-absis: add the number of chartlines*their height
                    (\firstweekendshift*\myxunit+\i*7*\myxunit+2*\myxunit,-\numbttitlelines*\myyunittitle -\numbtchartlines*\myyunitchart);
                % if the first day is a sunday, it is not grayed.
                % so draw a rectangle for the first day of the chart.
                \ifthenelse{\firstweekendshift=6}{
                    \fill [red!10]
                        (0,-\numbttitlelines*\myyunittitle)
                        rectangle
                        (1*\myxunit,-\numbttitlelines*\myyunittitle -\numbtchartlines*\myyunitchart);
                }{}
            \end{scope}
    \end{ganttchart}
\end{document}