%!TEX root = thesis.tex
\chapter*{Abstract}
\addcontentsline{toc}{chapter}{Abstract}
\thispagestyle{plain}

% The abstract is a short summary of the thesis. It announces in
% a brief and concise way the scientific goals, methods, and most important
% results. The chapter ``conclusions'' is not equivalent to the abstract!
% Nevertheless, the abstract may contain concluding remarks. The abstract
% should not be discursive. Hence, it cannot summarize all aspects of the thesis
% in very detail. Nothing should appear in an abstract that is not also
% covered in the body of the thesis itself. Hence, the abstract should be the
% last part of the thesis to be compiled by the author.

% A good abstract has the following properties: \emph{Comprehensive:} All major
% parts of the main text must also appear in the abstract. \emph{Precise:}
% Results, interpretations, and opinions must not differ from the ones in the main
% text. Avoid even subtle shifts in emphasis. \emph{Objective:} It may contain
% evaluative components, but it must not seem judgemental, even if the thesis
% topic raises controversial issues. \emph{Concise:} It should only contain the
% most important results. It should not exceed 300--500 words or about one page.
% \emph{Intelligible:} It should only contain widely-used terms. It should
% not contain equations and citations. Try to avoid symbols and acronyms (or at
% least explain them). \emph{Informative:} The reader should be able to quickly
% evaluate, whether or not the thesis is relevant for his/her work.

% An Example: The objective was to determine whether \dots (\emph{question/goal}).
% For this purpose, \dots was \dots (\emph{methodology}). It was found that \dots
% (\emph{results}). The results demonstrate that \dots (\emph{answer}).
\small
Global glacier models are imperative to project future glacier ice mass loss and the accompanied sea level rise. Glaciers are controlled by their mass balance and the resulting ice flow.
The scarcity of ice volume observations makes the calibration, and especially validation, of geometry models difficult. As a consequence, most global model use \vas{} relations instead of a dedicated ice physic model. Over the last years, the increasing accuracy and coverage of global data inventories, in combination with new and automated methods as well as increased computing power allowed for the development of more sophisticated models. This study compares the dynamic response of the global glacier model based on scaling relations by \citet{Marzeion2012b} to the globally applicable flowline model based on the Shallow Ice Approximation by \citet{Maussion2019} under controlled boundary conditions.

The re-implemented \vas{} model is able to correctly simulate glacier changes in response to various climatic forcings. The results are conceptually correct and qualitatively comparable to---but generally smaller than---those of the flowline model. The differences between \vas{} model and flowline model are most pronounced for idealized equilibrium experiments and can be attributed to a missing mass-balance-elevation feedback and the overestimation of model-internal time scales. Additionally, these too long time scales result in oscillating adjustments of glacier geometries to step changes in climate. Nevertheless, the \vas{} model's performance is still competitive for global applications and the introduction of ice physics modules does not completely devalue the scaling approach.

The results of ice volume change projections for Central Europe and High Mountain Asia over the the 21st century forced by CMIP6 data are also comparable to the literature, even though on the conservative side. While the values of ice volume change relative to 2020 may differ substantially between the \vas{} model and the flowline model, the values of absolute ice mass loss are in closer agreement. For example, the projected ice loss relative to 2020 for South Asia East (RGI region 15) ranges between $-41\SI{\pm4}{\percent} \equiv 0.93\SI{\pm0.10}{\milli\meter\sealevelrise}$ under SSP1-2.6 and $-64\SI{\pm5}{\percent} \equiv 1.44\SI{\pm0.10}{\milli\meter\sealevelrise}$ under SSP5-8.5 for the \vas{} model, and $-55\SI{\pm5}{\percent} \equiv 1.00\SI{\pm0.10}{\milli\meter\sealevelrise}$ under SSP1-2.6 and $-76\SI{\pm5}{\percent} \equiv 1.40\SI{\pm0.09}{\milli\meter\sealevelrise}$ under SSP5-8.5 for the flowline model. The values of both models are generally more comparable for High Mountain Asia (given the larger population size), while estimations of both models for Central Europe the can be outside each others uncertainty bounds (especially under lower forcings).



