%!TEX root = thesis.tex
\chapter{Methodology}\label{chap2}
\thispagestyle{plain}

This chapter provides a detailed description of the methodology. It is
sometimes called Experimental Section. Depending on the subject it is a
``synonym'', e.g., for Theoretical Section, Computational Methods, Model
Description and Setup, Field Work, and so on. Hence, this chapter contains a
description of \emph{what has been done} in order to address the scientific
question raised in the chapter Introduction. However, it does \emph{not} contain
the results! 


% ==== SECTION 1 ===============================================================
\section{Experimental Set-up}\label{2sec:1}
Depending on the topic of the science thesis, this chapter may contain a
description of the experimental set-up, the field experiment,
datasets, instruments, measurement procedures, analysis techniques, calibration
and quality control, and other things. In case of a modeling study it may
contain the formulation and derivation of model equations, the formulation of
initial and boundary conditions, the data used to drive and validate the model,
an overview of the model set-up (e.g., parameter set-up), modifications of the
``original'' model code, a description of relevant parameterizations,
a theoretical background needed for the interpretation of model results.


% ==== SECTION 2 ===============================================================
\section{Model Equations}\label{2sec:2}

\subsection{Subsection}
Use subsections to structure your thesis. The first and second component of the
momentum equation is shown in equation (\ref{2equ:1}) and (\ref{2equ:2}),
respectively. Together with (\ref{2equ:3}) they form the set of shallow-water
equations implemented in a numerical model.

\subsubsection{Subsubsection}
You can also use ``subsubsections''. However, they do not carry a separate
heading number and they do not appear in the Table of Contents.

\subsection{Equation}
As an example for the \verb|equation| environment, I show the equations used in
the numerical shallow-water model (SWM) developed by
\citet{scha93Aag,scha93Bag}:
% ---- equation 1:
\begin{equation}
\frac{D\hat{u}}{D\hat{t}}+\frac{\partial(\hat{h}+\hat{H})}
                               {\partial\hat{x}}=0,
\label{2equ:1}
\end{equation}
% ---- equation 2:
\begin{equation}
\frac{D\hat{v}}{D\hat{t}}+\frac{\partial(\hat{h}+\hat{H})}
                               {\partial\hat{y}}=0,
\label{2equ:2}
\end{equation}
% ---- equation 3:
\begin{equation}
\frac{\partial\hat{H}}{\partial\hat{t}}+\frac{\partial(\hat{u}\hat{H})}
                                             {\partial\hat{x}}
                                       +\frac{\partial(\hat{v}\hat{H})}
                                             {\partial\hat{y}}
                                                =0,
\label{2equ:3}
\end{equation}
with the non-dimensional variables (henceforth generally labelled with hats)
$\hat{u}$ and $\hat{v}$ as the two horizontal velocity components, $\hat{H}$ and
$\hat{h}$ as fluid layer depth and terrain height, respectively,
$\hat{Z}=\hat{h}+\hat{H}$ as fluid layer height, and $\hat{t}$ as time.
Equations~(\ref{2equ:1})--(\ref{2equ:3}) are non-dimensionalized with the
following scales: a typical length $L$ for the horizontal length scale, the
initial far-upstream depth of the fluid layer $H_{\infty}$ (with $h_{\infty}=0$)
for the vertical length scale, the phase speed of linear gravity waves
$\sqrt{g^*H_{\infty}}$ for the velocity scale, and the time scale
$L/\sqrt{g^*H_{\infty}}$.
