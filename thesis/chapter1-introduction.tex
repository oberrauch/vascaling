%!TEX root = thesis.tex
\chapter{Introduction}\label{chap:introduction}
\thispagestyle{plain}

% ====SECTION 1 ================================================================
\section{Motivation} % (fold)
\label{sec:motivation}

% The chapter Introduction leads the reader into the subject matter
% of the thesis. It is sometimes called Statement/Formulation/Definition/
% Presentation of the Problem. It may start with a so-called Motivation. It also
% contains the State of Knowledge or State of Research which is based on a
% literature survey (see section 2). Further, it contains the Scientific
% Questions and/or the Goals that are addressed in the main part of the thesis
% (see section 3). Finally, it provides an Outline of the science thesis (see
% end of section 3).

% section motivation (end)


% ==== SECTION 2 ===============================================================
\section{State of Research} % (fold)
\label{sec:state_of_research}

% section state_of_research (end)
\section{State of Research}\label{1sec:2}

% Based on the literature survey, the writer draws a picture of the existing
% knowledge in a specific field and points to open questions. Hence, after this
% survey the Introduction will ultimately culminate in the formulation of
% specific scientific questions/goals addressed in the thesis (see section 3).


% ==== SECTION 3 ===============================================================
\section{Goals and Outline}\label{1sec:3}

% After the literature survey the Introduction will ultimately culminate in the
% formulation of specific scientific questions, aims or goals. Hence, near the
% end of the Introduction there will often appear sentences like:
% - The goal of the investigation thus became trying to find out if ...
% - For this reason it appeared reasonable to attempt ...
% - It therefore became necessary to clarify whether ...
% See http://en.wikipedia.org/wiki/SMART_criteria as a guideline.


% Finally you should present an outline of your science thesis. Explain what the
% reader will find in the following chapters. For example, chapter 3 describes
% the methodology. The results are presented in chapter 3. A discussion is
% provided in chapter 4 and the conclusions are drawn in chapter 5.
