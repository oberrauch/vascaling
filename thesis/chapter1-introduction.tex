%!TEX root = thesis.tex
\chapter{Introduction}\label{chap:introduction}
\thispagestyle{plain}

% ====SECTION 1 ================================================================
\section{Motivation} % (fold)
\label{sec:motivation}

    % The chapter Introduction leads the reader into the subject matter
    % of the thesis. It is sometimes called Statement/Formulation/Definition/
    % Presentation of the Problem. It may start with a so-called Motivation. It also
    % contains the State of Knowledge or State of Research which is based on a
    % literature survey (see section 2). Further, it contains the Scientific
    % Questions and/or the Goals that are addressed in the main part of the thesis
    % (see section 3). Finally, it provides an Outline of the science thesis (see
    % end of section 3).

    Past and ongoing global warming is labeled as \emph{virtually certain}, \emph{unequivocal} and \emph{unprecedented} in the latest assessment report (AR5) of the Intergovernmental Panel on Climate Change \citep{IPCC2013}. 
    One of many effects of rising global air temperature is the increased melt of snow and ice in polar and high mountain regions. Glaciers and ice caps outside the major ice sheets in Greenland and Antarctica (hereafter simply referred to as \emph{glaciers}) make up less than \SI{1}{\percent} of the global land ice mass \citep[cf.][]{Farinotti2019, Vaughan2013}. However, the direct and indirect effects of changes in mountain cryosphere are disproportional and reach from the mountain regions themselves over the surrounding plains all the way to the oceans \citep{IPCC2019_TS}: destabilization of slopes by retreating glaciers and thawing permafrost affects the frequency and severity of geohazards \citep[e.g.,][]{Richardson2000, Deline2015, Haeberli2017}; changes in seasonal melt-water runoff impact water availability for consumption, agriculture and hydropower \citep[e.g.,][]{Farinotti2016,Huss2018, Ali2018, Immerzeel2020}; biodiversity and ecosystems are affected by changes in mountain habitats \citep[e.g.,][]{Milner2017}; tourism and mountain sports have to adapt to a decline in snow cover and glacier area \citep[e.g.,][]{Stewart2016, Lemieux2018, Mourey2019}. To cap it all, the contribution of melting glaciers to sea level rise is probably the most relevant effect on a global scale---yet possibly the least perceivable one.

    Despite their almost negligible share in global land ice volume, glaciers have contributed significantly to the observed sea level rise over the 20th century \citep[e.g.,][]{Marzeion2015, Bamber_2018}. Up to \SI{30}{\percent} of the current global sea level rise can be attributed to mass loss of glaciers, which is equivalent to the contribution of the Greenland ice sheet and even exceeds the contribution of the Antarctic ice sheet \citep{Zemp2019}. Thereby, most of the mass loss over the industrial era---if not all of it---can be attributed to anthropogenic forcings \citep{Marzeion2014a, Roe2020}, given that the central estimate of anthropogenic contribution to the industrial era temperature rise is basically \SI{100}{\percent} \citep{Allen2018}. Projections of future mass loss show a continued significant contribution of glaciers over the entire 21st century, even though the absolute amount is highly dependent on the chosen emission scenario \citep{Hock2019a, Marzeion2020}.

    As seen by their widespread effects, glaciers and their changes are integral parts of the climate system and therefore imperative to be observed and investigated. Estimations of past changes in global glacier ice mass are difficult due to the comparably scarce data availability. Numerical glacier models can improve the accuracy and precision of such estimates. But more importantly, glacier models are indispensable for global projections that go beyond simple extrapolations. Despite all that, there are currently only a handful of models capable of estimating future changes in global glacier ice mass as a response to climate input data. All except one were developed in the last ten years and most of them rely on volume/area scaling relations to represent glacier geometry changes \citep[e.g.,][]{VanDeWal2001,Marzeion2012b,Radic2014a}. Only the Open Global Glacier Model \citep[OGGM,][]{Maussion2019} implements a flowline model based on the shallow ice approximation \citep[SIA e.g.,][]{Hutter1983}. The following section provides an overview over the current state of research, including a brief history of global glacier modeling and the accompanied sea level rise projections.

% section motivation (end)


% ==== SECTION 2 ===============================================================
\section{A brief history of global glacier modeling} % (fold)
\label{sec:history_of_glacier_modeling}
    
    % Based on the literature survey, the writer draws a picture of the existing
    % knowledge in a specific field and points to open questions. Hence, after this
    % survey the Introduction will ultimately culminate in the formulation of
    % specific scientific questions/goals addressed in the thesis (see section 3).

    Computer models are widely used in all earth sciences, providing a valuable part to the scientific process and a viable addition to classical observations and experiments. In meteorology and climatology for example, general circulation models (GCM) date back to the 1960s (see \citep{Williamson2007} for an overview), and are imperative for daily weather forecast as well as for century-scale climate projections \citep[e.g.,][]{Cote2015, Lauritzen2011}. And while ice sheet models have a longstanding tradition in cryospheric research \citep[e.g.,][]{Pattyn2012, Nowicki2016}, the most glacier models---especially on a global scale---have emerged only over the past few years. This may be due to the lack of observational data when compared to atmospheric processes and/or the higher complexity of glacier ice dynamics when compared to ice sheets. The following section provides a brief overview over the evolution of global glacier models over the last twenty years. Hereby, the focus is on the general complexity of the dynamic model core, the spatial and temporal resolution, the resolved, parametrized and neglected processes, and the data used for calibration and validation. Additionally, the performance differences between temperature-index and energy balance mass balance models is noted, but not discussed in more depth. A good overview over all assessments of global glacier mass change (from in-situ observations over satellite gravimetry up to recent modeling approaches) can be found in \citet{Radic2014}.

    % Initial glacier models
    Numerical models for single glaciers date back to the late 20th century \citep[e.g.,][]{Budd1975, Bindschadler1982, Oerlemans1997}, while global applications were heavily hampered by a lack of observational data at the time (and to some degree they still are today). \citet{VanDeWal2001} pioneered the usage of \vas{} models for global sea level rise projections. Their model resolves fifteen glacier size classes within one-hundred individual glaciated regions and is forced by GCM data. Building on the mass balance sensitivity parametrization of \citet{Gregory1998}, yearly ice volume changes are computed for each size class within each region. Glacier area is then updated using the \vas{} relation for each region, depending on its characteristic size distribution. Incomplete inventories and a lack of observational data at the time of publication made assumptions and extrapolations necessary: the mass balance sensitivities to temperature and precipitation were based on observations of only twelve glaciers; the size distribution was known for and extrapolated from only forty-one of the one-hundred regions; In the end, only about 100\,000 individual glaciers were explicitly considered. Despite that, the contribution of glaciers to projected sea level rise over a simulation period of 70 years could be corrected down by \SI{19}{\percent} in comparison to prior estimates performed with constant glacier area and/or constant mass balance sensitivities.

    A similar approach to estimate global glacier volume change and accompanied sea level rise over the 21st century was used by \citet{Raper2006}, by combining a degree-day model \citep{Braithwaite2003} and a geometric model based on scaling relations \citep{Raper2000}. While the model had a higher spatial resolution and ran on a $\SI{1}{\degree}\times\SI{1}{\degree}$ grid, the major problem at the time was still estimating the needed parameters for areas not sufficiently covered by inventories. The mass balance model could run only in seven regions who had sufficient data coverage. A multiple linear regression (with annual precipitation and summer temperatures as explanatory variables and mass balance gradient as dependent variable) was computed from the mass balance profiles of each single grid cell of these seven regions. This regression was then used to extrapolate the results to all other regions. Statistical glacier characteristics like volume and area distribution, area-elevation distribution and elevation range were also estimated from those seven regions and extrapolated into each grid cell. In addition to the aforementioned assumptions and extrapolations, the coarse grid creates multiple problems for the \vas{} model. On one hand, multiple glaciers are combined in a single cell, i.e., treated as glacier complexes, and on the other hand, single glaciers are subdivided along grid cell boundaries. According to \citet[Section 8.6]{Bahr2015}, \vas{} should not be applied to glacier complexes, nor to fractional parts of glaciers.

    The IPCC's fourth assessment report \citep{IPCC2007} estimated the sea level contribution of glaciers excluding any dynamic effects. The estimation was based on a consensus estimate by \citet{Kaser2006}, who in turn summarized three different publications which all used the same data. The use of a constant mass balance sensitivity results in a complete melt of glaciers for any warming, since no new equilibrium condition can be established \citep{Raper2006, Pfeffer2008}. Frontal ablation of water-terminating glaciers was not considered, despite its considerably fast ice discharge of ice directly into the ocean, which was obviously already known at the time \citep{Pfeffer2008}. Additionally, glaciers and ice caps on the periphery of the Greenland and Antarctic ice shield were still missing from inventories. Their respective sea level rise contribution was estimated by blatantly adding \SI{20}{\percent} to the global estimate. The general reasoning was that the relevant processes were not yet well enough understood to allow for reliable model estimates. In response, some studies circumvented the problems of direct modeling by e.g., extrapolating observed mass loss rates and their acceleration over the last decades \citep{Meier2007}, predefining low and high sea level rise projections and investigate the plausibility of climatic scenarios needed to meet them \citep{Pfeffer2008}, or projecting observed accumulation-area ratios (AAR) and their rate of change into the future \citep{Bahr2009}.

    Motivated by the discrepancies of the few 21st century projections of global mass loss of glaciers, \citet{Radic2011} published projections from a new model, which resolves each glacier individually. However, the scarcity of usable mass balance observations still hampered the model performance. The proposed calibration of the elevation dependent surface mass balance model with monthly temperature reanalysis and precipitation climatologies relied on observations of seasonal mass balance profiles of only thirty-six glaciers. Transfer functions based on a multiple linear regression were used to extrapolate the resulting mass balance calibration parameters to all glaciers in the global inventory. At the time this meant only about 120\,000 glaciers and 2600 ice caps, corresponding to about \SI{40}{\percent} of the total global glacier area. Area-averaged mass balance estimates for forty-one glacier regions (based on observations from only over 300 glaciers) were used to tune the model by an iterative process, which impeded an independent model validation with 20th century observations. The same area-averaged mass balance estimates were then used for initialization. The calibrated an initialized model ran with downscaled monthly temperature and precipitation projections of ten different GCMs, based on a mid-range emission scenario. Volume/area and volume/length scaling was used to update glacier geometries and allowed for a potential new equilibrium under a changed climate. While modeling each individual glacier was a step in the right direction there was still room for improvements, especially concerning the calibration and validation process.

    \citet{Marzeion2012b} described the construction, validation and application of a another new global surface mass balance model. This model will is implemented and discussed in detail in this study.The model includes a simple representation of glacier geometry change, based on a combination of volume/length, volume/area and response time scaling. It is able to simulate past and future glacier states using observed climate data, model based climate reconstruction and GCM projections. It is the first global glacier model that was independently validated, using a leave-one-glacier-out cross validation of all 255 globally available glaciers with sufficiently long mass balance records. Thanks to the back then new inventory of global glacier outlines---the first version of the Randolph Glacier Inventory \citep[RGI][]{Arendt2012_RGIv1.0}---each individual glacier and ice cap outside Antarctica could be modeled. The problem arising from the undersampling of mass balance observation was circumvented by a new calibration process: it is assumed that neighboring glaciers are most likely in an equilibrium state around the same time, while mass balance sensitivities of neighboring glaciers can be drastically different (due to different aspect, slope, shading, etc.). This means that instead of directly interpolating the mass balance sensitivities, the \textit{equilibrium years} are interpolated from the ten closest glaciers with mass balance observations (for details see Section~\ref{ssub:mb_calib}). The benefits of this technique are reflected in a decreased mass balance error, estimated by a leave-one-glacier-out cross validation. %TODO: summary and segue

    The latest IPCC assessment report \citep[AR5, Chapter 13]{Church2013} discusses the following 21st century sea level rise projections for global glaciers under different emission scenarios: \citet{Slangen2011} use an updated version of the model by \citet{VanDeWal2001} and provide additional uncertainty estimations for different sources; \citet{Marzeion2012b} as presented above; \citet{Giesen2013} also use a \vas{} model to account for geometry changes but include a simplified energy balance model with hourly resolution. However, only 89 glaciers are modeled explicitly (again due to missing input data) and the results are then upscaled to all other glacier grater than \SI{0.1}{\square\kilo\meter}; \citet{Radic2014a} use an updated version of the model by \citet{Radic2011} calibrated and validated with new data and applied to each individual glacier. The globally complete inventory of glacier location, area and hypsometry provided by the RGI drastically improved the projections made by all the above mentioned models. However, they all still rely on scaling relation to update glacier geometries and allow for new equilibrium states. Additionally, none of the models accounts for calving or other rapid dynamic responses.

    The increasing number of large scale datasets for all global glaciers over the last years---e.g., a satellite-borne digital elevation model \citep{Jarvis2008}, a distributed ice thickness and volume estimate \citep{Huss2012}, a consensus estimate of regionally distributed mass budget between 2003 and 2009 \citep{Gardner2013}, digital glacier outlines and areas \citep[RGI v4.0,][]{Pfeffer2014}---allows for new and more sophisticated models. The Global Glacier Evolution Model \citep[GloGEM,][]{Huss2015} computes each glacier's surface mass balance (including refreezing) with a temperature-index model on \SI{10}{\meter} elevation bands with monthly resolution, and frontal ablation of water-terminating glaciers with yearly resolution. The dynamic response of glacier geometry based on \citet{Huss2010} adjusts area, thickness and surface elevation of each elevation band. Thereby, a non-dimensional empirical function redistributes the total mass change to each individual band, assuming maximum ice thickness change at the glacier terminus and no thickness change at the glacier top. The parametrization of this so called $\Delta h$ function depends on the glacier size class. Glacier advance is accounted for by adding bands, if the thickness change exceeds a threshold. If the ice thickness of the lowest elevation band shrinks to zero, the band gets removed and the glacier retreats. Additionally, the computed sea level equivalent accounts for ice volume already below sea level, which systematically reduces the contribution of glaciers by about 11--\SI{14}{\percent}.

    The IPPC's most recent Special Report on the Ocean and the Cryosphere in a Changing Climate \citep[SROCC][]{IPCC2019_TS} includes updates of previous studies using new inventories and/or climate projections \citep{Bliss2014,Slangen2017,Hock2019a}. Additionally, the following two new models are included: GloGEM as described above, and HYOGA2 \citet{Hirabayashi2013}. HYOGA2 uses a temperature-index mass balance model and relies on volume/length scaling to update glacier geometries (only the area at the lowest elevation band is updated). As all IPCC report, SROCC does not provide much new information, but much rather summarizes established literature and provides an updated uncertainty analysis by adding new and updated models.

    The newest addition to the global glacier models is the OGGM \citep{Maussion2019}. The implemented temperature-index model and mass balance calibration is adapted from \citet{Marzeion2012b}, an added calving parametrization accounts for frontal ablation of water-terminating glaciers. Individual glaciers are represented by a depth integrated flowline, with the possibility of tributaries. Ice flow is computed by an explicit ice dynamics module, which solves the SIA equations on a staggered grid without converting them into a diffusion equation \citep{Maussion2019}. By doing so, the OGGM is the first and so far only globally applicable model using flowline representations of glaciers. 
    But much more importantly, it is open source (see the GitHub repository \url{https://github.com/OGGM/oggm}) and hence by design modular. This allows to compare the performance of various parameterizations of e.g., the mass balance model or downscaling procedures as well as the implementation of additional features to resolve new processes like e.g., the effect of debris cover. It is also a basic prerequisite for this study, since it enables the implementation of the original \citet{Marzeion2012b} model into the OGGM framework.

    Research involving global glacier models has come a long way in the last twenty years. In analogy to the Coupled Model Intercomparison Project \citep[CMIP, e.g.,][]{Eyring2016_CMIP} for atmospheric models, recent coordinated efforts for glacier models have been made, e.g., the Ice Thickness Models Intercomparison eXperiment \citep[\url{https://cryosphericsciences.org/activities/ice-thickness/}]{Farinotti2017, Farinotti2020} and the Glacier Model Intercomparison Project (GlacierMIP, \url{https://www.climate-cryosphere.org/mips/glaciermip}). Two recent studies on global sea level rise projections of glaciers are published as part of GlacierMIP. \citet{Hock2019a} provide a consensus estimate of glacier contribution to 21st century sea level rise projections. The results are based on the five global or nearly global (i.e., without Greenland and/or Antarctic periphery) models \citep{Slangen2012, Marzeion2012b, Giesen2013, Hirabayashi2013, Radic2014a, Huss2015} forced by multiple GMCs and emission scenarios. \citet{Marzeion2020} use four of those models and add the global model GLIMB \citep{Sakai2017}, the two nearly global models JULES \citep{Shannon2019} and OGGM, and four regional models to the ensemble to provide an updated uncertainty analysis. The new additional feature of both GLIMB and JULES is the implementation of an energy balance model to compute the surface mass balance (refer to the original publications for details about the individual models not discussed here). Seven of these thirteen models rely on some form of scaling relation to update glacier geometries while the OGGM is the only global model using a flowline model. It is therefore fair to say, that \vas{} based models were, and still are, the most broadly used type of models to estimate glacier geometry changes. Given their long tradition and broad application, several studies have investigated the performance, strength and weaknesses of scaling based models. An overview is given in the following paragraphs.

    \Vas{} is a simple power law relation $ V = c\,A^\gamma$, based on the empirical relationship between glacier volume $V$ and surface area $A$. The scaling constant $c$ is a random variable depending on glacier characteristics and needs calibration. The scaling exponent, $\gamma_\text{glacier} = 1.375$ for glaciers and $\gamma_\text{ice cap} = 1.25$ for ice caps, is a physically based constant and should not be changed \citep{Bahr1997b}. It can be formally derived by a dimensional analysis of the relevant continuum equations. The derivation does not rely on any assumptions like steady state conditions, perfect plasticity, plain strain or shallow-ice approximation, which is why \citet{Bahr2015} insist on the validity and viability of volume/area and similar scaling relations. Additionally, \vas{} has inherent benefits in dealing with the non-uniqueness and instabilities arising from the ill-posed nature of ice thickness inversions \citep{Bahr2014}. However, dimensional analysis do---by definition---only provide characteristic results, which should not be misunderstood as specific results for any specific state of the system. Such misconceptions and the accompanied successful and unsuccessful application of glacier scaling relations have lead to some controversy in past publications \citep[][see Table 1 for an overview]{Bahr2015}.

    \citet{Radic2007} used a set of thirty-seven synthetic glaciers created by a one-dimensional flowline model to compute and compare scaling exponents between steady state and transient conditions. There are several problems with this approach, the first of which is that their estimated range for $\gamma$ is well outside any physically sensible bounds. Secondly, no assumption of steady state conditions is used for the derivation of the scaling relations. \Vas{} is fully valid for transient conditions, however it must be used in conjuncture with an appropriate response time scaling. Thirdly, the scaling exponent is a physically based constant and it's value does not change in time or space \citep{Bahr2015}. And lastly, the used sample size is rather small. This undersampling problem is even more severe in \citet{Radic2008}, where the performance of a \vas{} model for mass change projections of only six different real world glaciers is compared to a flowline model. Scaling relations are statistically relevant for a large enough population of glaciers, but do not apply to individual glaciers. Volume scaled from the area of a single glaciers can only be treated as order of magnitude estimate, the relative volume error is directly proportional to the relative error in scaling constant $c$ \citep{Bahr2015}. However, both publications endorse the use of \vas{} for glacier volume projections and sea level rise estimations. Results from \vas{} agree well with the flowline model, even though the ice loss is underestimated up to \SI{47}{\percent}. While initial volume estimations are highly sensitive to the chosen scaling parameters, volume projections over a century timescale are not, especially when normalized with initial values \citep{Radic2007, Radic2008}. \citet{Slangen2011} came to the same conclusion, by varying only $c$ and fixing $\gamma$ at appropriate values. More important are uncertainties introduced by the mass balance sensitivity, glacier inventory and emission scenario.

    \citet{Adhikari2012} performed a similar analysis, evaluating the \vas{} model with respect to a numerical model. Using an ensemble of 280 glaciers randomly generated by a three-dimensional Stokes model, they assessed the influence of different characteristics of topography, climate, glacier flow and disequilibrium with climate on the scaling relations. Most of the above mentioned points of criticism (wrongful steady state assumption, non-constant scaling exponent) also applies to this study. However, their estimated scaling exponents (for steady state and transient conditions) are much more in line with the theoretical values and past observations, most likely due to the bigger sample size. \citet{Farinotti2013} show that in order to estimate the total ice volume of a population of more than thousand glaciers within \SI{30}{\percent} of its true value, 280 pairs of volume and area measurements are needed to calibrate the scaling parameters. Hereby, reducing either the population size or the number of measurements for calibration decreases the estimation's accuracy. Uncertainties in ice volume measurements are negligible, if a systematic bias can be ruled out.

    None of the above mentioned studies evaluated the performance of scaling models on the example of real world glaciers on a regional scale. The modular design of the OGGM allows the implementation of the original \citet{Marzeion2012b} model in the same framework and facilitates an easy head-to-head comparison.


% section history_of_glacier_modeling (end)

% ==== SECTION 3 ===============================================================
\section{Goals and Outline} % (fold)
\label{sec:goals_and_outline}

    % After the literature survey the Introduction will ultimately culminate in the
    % formulation of specific scientific questions, aims or goals. Hence, near the
    % end of the Introduction there will often appear sentences like:
    % - The goal of the investigation thus became trying to find out if ...
    % - For this reason it appeared reasonable to attempt ...
    % - It therefore became necessary to clarify whether ...
    % See http://en.wikipedia.org/wiki/SMART_criteria as a guideline.

    As outlined above, \vas{} has a long academic tradition. At the time, it was used with great success in many global glacier models. Nowadays, new global inventories allow for more sophisticated glacier models on a global scale. The main goal of this thesis is to establish the benefits of the OGGM flowline model over its simpler \vas{} predecessor. 
    This is done by comparing both models head-to-head in a variety of different experiment with a two part focus:
    \begin{enumerate*}[label=(\alph*)]
        \item investigating the \vas{} model for its strength and test whether its limited geometry representation might suffices for projections on a regional scale, and
        \item identifying relevant physical processes that are resolved by the flowline model but not the \vas{} model.
    \end{enumerate*}
    The question about the flowline models accuracy (validated against observational data) will not be answered here.
    % Finally you should present an outline of your science thesis. Explain what the
    % reader will find in the following chapters. For example, chapter 3 describes
    % the methodology. The results are presented in chapter 3. A discussion is
    % provided in chapter 4 and the conclusions are drawn in chapter 5.
    
    % Chapter 2
    While the model implementation may be seen as a pure necessity, it is actually a central part of this work and therefore detailed in Chapter~2. The implementation of the \vas{} model into the OGGM framework highlights the possibilities arising from a modular, community based glacier model. The programming related implementation details provided here can be seen as a basic form of documentation. The chapter starts with the general theoretical concepts about the glacier \vas{} relations, the used temperature-index mass balance model, and the scaling glacier evolution model. Additionally, the experimental setup for all following analyses is detailed.
    % Chapter 3
    The experiments shown in Chapter~3 start with a single glacier test case, including a time series analysis. Since \vas{} should only be applied to large populations of glaciers, the same experiments are also performed on a regional scale, by investigating the aggregate ice volume of all Alpine glaciers. After identifying problems and potential tuning parameters, the model's sensitivity to its scaling parameters is tested. This all culminates in a final projection of mass change for all Alpine glaciers under different warming scenarios.
    % Chapter 4 + 5
    Chapter~4 provides a discussion of the identified strength and weaknesses of each model, by relating the result to recently published literature. A final summary and an outlook on possible future work can be found in Chapter~5.

% section goals_and_outline (end)
