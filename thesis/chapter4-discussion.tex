%!TEX root = thesis.tex
\chapter{Discussion}\label{chap:discussion}
\thispagestyle{plain}

% The discussion is the interpretation and evaluation of the results. It is a
% comparison of your results with previous findings. It provides the answer to
% the scientific questions raised in the introduction. It is the "nerve center"
% of a thesis, whereas the chapter Results may be seen as the "heart".

% Clearly separate between your own contributions and those of others. Provide
% rigorous citations of appropriate sources! Explicitly refer to specific 
% results presented earlier. A certain amount of repetition is necessary. Order
% discussion items not chronologically but rather logically.

% The chapter Results answers the question: What has been found? (Facts). The
% chapter Discussion answers the question: How has the  result to be interpreted?
% (Opinion).

% The most important message should appear in the first paragraph. The answer to
% the key question may appear in the first sentence: e.g., did your original idea
% work, or didn't it? The following questions may be answered in the discussion
% section:
% - Why is the presented method simpler, better, more reliable than previous
% ones?
% - What are its strengths and its limitations?
% - How significant are the results?
% - How trustworthy are the observations?
% - Under which conditions and for which region are the results/method valid?
% - Can the results be easily transferred to other regions or fields?

