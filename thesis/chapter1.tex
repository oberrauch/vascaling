%!TEX root = thesis.tex
\chapter{Introduction}\label{chap1}
\thispagestyle{plain}

% ====SECTION 1 ================================================================
\section{Motivation}
The chapter Introduction leads the reader into the subject matter
of the thesis. It is sometimes called
Statement/Formulation/Definition/Presentation of the Problem. It may start with
a so-called Motivation. It also contains the State of Knowledge or State
of Research which is based on a literature survey (see section \ref{1sec:2}).
Further, it contains the Scientific Questions and/or the Goals that are
addressed in the main part of the thesis (see section \ref{1sec:3}). Finally,
it provides an Outline of the science thesis (see end of section \ref{1sec:3}).


% ==== SECTION 2 ===============================================================
\section{State of Research}\label{1sec:2}

Based on the literature survey, the writer draws a picture of the existing
knowledge in a specific field and points to open questions. Hence, after this
survey the Introduction will ultimately culminate in the formulation of specific
scientific questions/goals addressed in the thesis (see section \ref{1sec:3}).

To cite a certain source (e.g., a paper) use the citation commands
\verb|\citet| and \verb|\citep| of the \verb|natbib|
package.\footnote{
\url{http://www.ctan.org/tex-archive/macros/latex/contrib/natbib/natnotes.pdf}}
Together with the bibliography style \verb|ametsoc.bst|, which is
included in the \LaTeX{} manuscript template for AMS journals\footnote{
\url{http://www.ametsoc.org/PUBS/journals/AMS_Latex_V3.0.tar.gz}}, \verb|natbib|
produces citations in the author-date format together with a list of
references that fulfill the AMS citation standard.\footnote{
\url{http://www.ametsoc.org/PUBS/journals/author_reference_guide.pdf}}

As an example, you can cite papers like \citet{hann66Aag} and \citet{scha93Aag}
which have to be specified in your BibTeX database file (in this case it is
\verb|mybibfile.bib|). More than one article of the same author can be cited
like here: \citet{hoin85Aag,hoin90Aag} studied foehn winds.

You may want to split your review of the literature into several sections.
Further, use paragraphs to structure your introduction. If you like to cite
papers in brackets (\emph{passive citations}) you can do this
as in the following sentence: Gap flows have been studied in the Strait
of Gibraltar \citep{scor52Aag,dorm95Aag}, in the French Rh\^one Valley
\citep{pett82Aag}, near Hokkaid\=o in Japan \citep{arak69Aag}, near
Unimak Island in the Aleutian Chain \citep{pan-99Aag}, and in the Howe
Sound of British Columbia \citep{jack94Aag,jack94Bag}. Citation of a
Dissertation: The gap flow in the Wipp Valley has been studied by
\citet{gohm03Aag}. Citation of a conference paper: \citet{gohm06Aag}
investigated the boundary layer structure in the Inn Valley. Citation of an
online document: The AMS provides a guideline for preparing citations and
references \citep{ams-09Aag}.


% ==== SECTION 3 ===============================================================
\section{Goals and Outline}\label{1sec:3}

After the literature survey the Introduction will ultimately culminate in the
formulation of specific scientific questions, aims or \emph{goals}. Hence, near
the end of the Introduction there will often appear sentences like:\\[1ex]
\dots The goal of the investigation thus became trying to find out if \dots\\
\dots For this reason it appeared reasonable to attempt \dots\\
\dots It therefore became necessary to clarify whether \dots\\ 

Instead of describing the goals in one paragraph, you may want to structure them
with the \verb|itemize| command:
\begin{itemize}
\item[(1)] First goal.
\item[(2)] Second goal.
\item[(3)] Third goal.
\end{itemize}

The so-called SMART
criteria\footnote{\url{http://en.wikipedia.org/wiki/SMART_criteria}} might be
used as a guideline to define reasonable goals. Here, the acronym SMART
describes the properties of ``good'' goals: \underline{s}pecific,
\underline{m}easurable, \underline{a}ttainable, \underline{r}elevant,
\underline{t}ime-bound.

The introduction may also describe briefly the methodology chosen and the
materials (e.g., data, instruments, etc.) used. However, a detailed description
will follow in the main part (see chapter \ref{chap2}).

Finally you should present an \emph{outline} of your science thesis. Explain
what the reader will find in the following chapters. For example, chapter
\ref{chap2} describes the methodology. The results are presented
in chapter \ref{chap3}. A discussion is provided in chapter \ref{disc} and the
conclusions are drawn in chapter \ref{concl}.
