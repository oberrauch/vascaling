% unused snippets

% Table showing the e-folding time scales for Hintereisferner, time scales as rows
% --------------------------------------------------------------------------------

\begin{table}[htp]
  \centering
  \ra{1.4}
  \caption{e-folding time scales for Hintereisferner (RGI60-11.00897) in response to a step change in climate of $\Delta T = $\SI{\pm0.5}{\celsius} relative to the average climate between 1912 and 1942. Time scales are computed for changes in ice volume, surface area and glacier length, denoted as $\tau_V$, $\tau_A$ and $\tau_L$, respectively. \textit{V/A scaling} refers to the \vas{} model, while \textit{Flowline} refers to the Open Global Glacier Model (OGGM).}
  \label{tab:hintereisferner_time_scales}
  \begin{tabular}{@{}llccccc@{}}
    \toprule
    {} & \phantom{.} & \multicolumn{2}{c}{\textbf{V/A scaling}} & \phantom{ab} & \multicolumn{2}{c}{\textbf{Flowline}} \\
    \cmidrule{3-4}\cmidrule{6-7}
    \textbf{Time scales} & & \SI{-0.5}{\celsius} & \SI{+0.5}{\celsius} & & \SI{-0.5}{\celsius} & \SI{+0.5}{\celsius} \\
    \midrule
    $\bm{\tau_V}$ \textbf{[\si{\year}]} & & 39 & 36 & & 139 & 79 \\
    $\bm{\tau_A}$ \textbf{[\si{\year}]} & & 57 & 52 & & 159 & 107 \\
    $\bm{\tau_L}$ \textbf{[\si{\year}]} & & 85 & 80 & & 174 & 123 \\
    \bottomrule
  \end{tabular}
\end{table}

% Table showing the Hintereisferner equilibrium values, geometries as rows, no intital values
% -------------------------------------------------------------------------------------------

\begin{table}[htp]
  \centering
  \ra{1.4}
  \caption{Hintereisferner (RGI60-11.00897) equilibrium values after 1000 years of model evolution under a constant equilibrium climate with a temperature bias of \SI{-0.5}{\celsius} and \SI{+0.5}{\celsius}, respectively. Percentage values in parenthesis indicate normalized values in respective to their initial values.}
  \label{tab:hintereisferner_equilibrium_values}
  \begin{tabular}{@{}lrlcrlcrlcrl@{}}
    \toprule
    {} & \multicolumn{5}{c}{$\bm{\Delta T}$\textbf{ = \SI{-0.5}{\celsius}}} & \phantom{a} & \multicolumn{5}{c}{$\bm{\Delta T}$\textbf{ = \SI{+0.5}{\celsius}}} \\
    \cmidrule{2-6} \cmidrule{8-12}

    {} & \multicolumn{2}{c}{V/A scaling} & \phantom {} & \multicolumn{2}{c}{Flowline} & \phantom{a} & \multicolumn{2}{c}{V/A scaling} & \phantom {} & \multicolumn{2}{c}{Flowline} \\
    \midrule
    \textbf{Volume [\si{\cubic\kilo\meter}]} &  0.70 & (117\%) & \phantom {} &  1.37 & (171\%) & \phantom{a} &  0.50 & (84\%) & \phantom {} &  0.47 & (58\%) \\
    \textbf{Area [\si{\square\kilo\meter}]} &  9.02 & (112\%) & \phantom {} &  10.68 & (133\%) & \phantom{a} &  7.08 & (88\%) & \phantom {} &  6.17 & (77\%) \\
    \textbf{Length [\si{\kilo\meter}]} &  5.26 & (107\%) & \phantom {} &  9.95 & (144\%) & \phantom{a} &  4.52 & (92\%) & \phantom {} &  4.20 & (61\%) \\
    \bottomrule
  \end{tabular}
\end{table}

% Table showing the Hintereisferner equilibrium values, geometries as rows, vertical version
% ------------------------------------------------------------------------------------------

\begin{sidewaystable}[htp]
  \centering
  \ra{1.4}
  \caption{Hintereisferner (RGI60-11.00897) equilibrium values after 1000 years of model evolution under a constant equilibrium climate with a temperature bias of \SI{-0.5}{\celsius} and \SI{+0.5}{\celsius}, respectively. Percentage values in parenthesis indicate normalized values in respective to their initial values.}
  \label{tab:hintereisferner_equilibrium_values}
  \begin{tabular}{@{}lcccrlcrlcrlcrl@{}}
    \toprule
    % first level header
    {} & \multicolumn{2}{c}{\textbf{Initial values}} & \phantom{asdf} & \multicolumn{5}{c}{$\bm{\Delta T}$\textbf{ = \SI{-0.5}{\celsius}}} & \phantom{a} & \multicolumn{5}{c}{$\bm{\Delta T}$\textbf{ = \SI{+0.5}{\celsius}}} \\
    % second level header
    \cmidrule{2-3} \cmidrule{5-9} \cmidrule{11-15}
    {} & V/A scaling & Flowline & \phantom {} & \multicolumn{2}{c}{V/A scaling} & \phantom {} & \multicolumn{2}{c}{Flowline} & \phantom{a} & \multicolumn{2}{c}{V/A scaling} & \phantom {} & \multicolumn{2}{c}{Flowline} \\
    % table body
    % volume
    \midrule
    \textbf{Volume [\si{\cubic\kilo\meter}]} & 0.60 & 0.80 & &  0.70 & (117\%) & \phantom {} &  1.37 & (171\%) & \phantom{a} &  0.50 & (84\%) & \phantom {} &  0.47 & (58\%) \\
    % area
    \textbf{Area [\si{\square\kilo\meter}]} & 8.04 & 8.04 & &  9.02 & (112\%) & \phantom {} &  10.68 & (133\%) & \phantom{a} &  7.08 & (88\%) & \phantom {} &  6.17 & (77\%) \\
    % length
    \textbf{Length [\si{\kilo\meter}]} & 4.89 & 6.90 & & 5.26 & (107\%) & \phantom {} &  9.95 & (144\%) & \phantom{a} &  4.52 & (92\%) & \phantom {} &  4.20 & (61\%) \\
    \bottomrule
  \end{tabular}
\end{sidewaystable}

Text with all the equilibrium values contained in the table, don't know if to inlcude since it looks/reads weird.
The equilibrium ice volume for the positive and negative mass balance scenario is \SI{0.70}{\cubic\kilo\meter} (\SI{117}{\percent}) and \SI{0.50}{\cubic\kilo\meter} (\SI{84}{\percent}) for the \vas{} model and \SI{1.37}{\cubic\kilo\meter} (\SI{171}{\percent}) and \SI{0.47}{\cubic\kilo\meter} (\SI{58}{\percent}) for the flowline model, respectively. The values in parenthesis are normalized with the respective initial values. The equilibrium surface area for the positive and negative mass balance scenario is \SI{9.02}{\squared\kilo\meter} (\SI{112}{\percent}) and \SI{7.08}{\squared\kilo\meter} (\SI{88}{\percent}) for the \vas{} model and \SI{10.68}{\squared\kilo\meter} (\SI{133}{\percent}) and \SI{6.17}{\squared\kilo\meter} (\SI{77}{\percent}) for the flowline model, respectively. The equilibrium length for the positive and negative mass balance scenario is \SI{5.26}{\kilo\meter} (\SI{107}{\percent}) and \SI{4.52}{\kilo\meter} (\SI{92}{\percent}) for the \vas{} model and \SI{9.95}{\kilo\meter} (\SI{144}{\percent}) and \SI{4.20}{\kilo\meter} (\SI{61}{\percent}) for the flowline model, respectively.

