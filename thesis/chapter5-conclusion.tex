%!TEX root = thesis.tex
\chapter{Conclusions}\label{chap:conclusions}
\thispagestyle{plain}

% This chapter contains consequences that derive from your results. It may also
% contain speculations. It may provide suggestions for future studies. Hence,
% the conclusions may provide an outlook and list open questions. Sometimes this
% chapter is part of the discussion. In such a case, the chapter reads
% "Discussion and Conclusions".

Glaciers are controlled by their mass balance and the resulting ice flow. Hence, a glacier model has to incorporate a mass balance model, to account for accumulation and ablation, and a dynamic model, to update glacier geometries in response to mass balance changes. A dynamic model based on scaling relations, such as the here presented combination of volume/area, volume/length and response time scaling, can produce qualitatively comparable results to a state-of-the-art flowline model. However, quantitatively speaking there large discrepancies between the \vas{} model and the flowline model.

The overall behavior of the \vas{} model is as expected, i.e., the model correctly simulates advances and retreats in response to step changes in climate as well as to natural climate variability, it has the ability to reach a new equilibrium, and it shows characteristics of a low-pass filter. However, changes in ice volume, surface area and glacier length are generally underestimated when compared to the flowline model. The model internal representation of time scales has been identified as the main source of those differences and other unexpected and unwanted behaviors. Based on the conducted sensitivity analysis, and backed by \citet{Roe2020}, we conclude that the used time scale parameterization results in too high values. The timescales could be exploited as future tuning parameter. Using the geometric timescales based on the terminus ablation rate \citep{Johannesson1989} rather than the turnover may improve the results.

Additional errors are introduced by missing physically relevant processes, which are not resolved by the scaling model. Those include but are not limited to, a missing mass-balance-elevation feedback, a constant average slope, and no awareness of the surrounding topography. While these processes may be secondary on a global scale, the regional estimates vary significantly between scaling and flowline model. The usage of more complex models is advisable for future projections of glacier mass changes, especially given the nowadays available data from constantly improving global inventories. However, the here provided theoretical comparison between the scaling and flowline model does not asses the model accuracy, since no validation against observational data is performed.

This study detailed the underlying principles and implementation process of a simple global glacier model based on glacier scaling relations \citep{Marzeion2012b} in the OGGM framework. This successful implementation, even if applicable only to the European Alps so far, shows the benefits of an open source, modular, community driven glacier model, and is only the first step in comparing different parameterizations. The global implementation of the scaling model and the implementation of additional geometry parameterizations \citep[e.g., GloGEM by][]{Huss2015} could give further insights into the benefits and limitations of explicit glacier dynamics.

